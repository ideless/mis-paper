\documentclass{article}

\usepackage{xcolor}
\usepackage{natbib}

\newcommand{\DescItem}[2]{\item[{#1}] \hfill \textcolor{blue}{#2}\\}
\newcommand{\Keywords}[1]{\\ \#\textit{#1}}

\title{Recent progress}
\author{Rui Mao}
\date{\today}

\begin{document}

\maketitle

\begin{description}
	\DescItem{\cite{pichler2018quantum}}{}
	The first proposal to use Rydberg atom arrays to solve MIS.

	\DescItem{\cite{pichler2018computational}}{}
	Showed that finding the ground state of Rydberg system is NP-complete.

	\DescItem{\cite{kim2022rydberg} Nature Physics}{3D}
	An early work that demonstrates solving MIS on non-planar graph and high-degree graph using 3D Rydberg atom arrays.

	\DescItem{\cite{cain2023quantum}}{}
	Introduce an additional local Hamiltonian to achieve a quadratic speedup over a wide range of classical algorithms.

	\DescItem{\textcolor{red}{*}\cite{dalyac2024graph}}{Review}
	Review of graph problems on atom arrays.
	It also gives brief discussion on recent hardware progress.

	\DescItem{\cite{wurtz2024industry}}{Applications}
	Industry applications of neutral-atom quantum computing solving MIS.

	\DescItem{\cite{byun2022finding} PRX Quantum}{Platonic graphs}
	Finding the MIS of platonic graphs using Rydberg atoms.

	\DescItem{\textcolor{red}{*}\cite{biedl2006three}}{VLSI, 3D}
	The volume of drawing a graph in 3D space is $\Theta(m\sqrt{n})$.

	\DescItem{\textcolor{red}{*}\cite{dalyac2023exploring}}{3D}
	Introduce the VLSI techniques to give a $O(n^{3/2})$ scheme for bounded degree graphs.

	\DescItem{\textcolor{red}{*}\cite{dell2010satisfiability}}{}
	Proved that unless PH collapes, there is no efficient and effective sparsification algorithm for SAT and vertex cover.
	SAT: no polynomial-time algorithm can compress a $n$-variable $d$-CNF formula to $O(n^{d-\epsilon})$ bits without loss of information.
	Vertex cover: no kernels consisting of $O(k^{2-\epsilon})$ edges.
	\textbf{TODO:} This paper possibly rules out sparsification algorithm for MIS, but I need to understand what ``kernel'' means.

	\DescItem{\cite{barredo2018synthetic}}{3D, experiment}
	Demonstrated the realization of fully loaded, arbitrarily-shaped 3D arrays with up to 72 atoms.
	The method is to first load enough atoms, then move them to the correct positions in a plane-by-plane manner.
	The move in XY plane in performed by AOD, while the move in Z axis in by tuning the focal length of some lens.
\end{description}

\bibliographystyle{plainnat}
\bibliography{refs}

\end{document}
