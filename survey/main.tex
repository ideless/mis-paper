\documentclass{article}

\usepackage{natbib}

\newcommand{\CiteDesc}[2]{\item[\cite{#1} #2] \hfill \\}
\newcommand{\Keywords}[1]{\\ \#\textit{#1}}

\begin{document}

\begin{description}
  \CiteDesc{pichler2018quantum}{arXiv}
  The first proposal to use Rydberg atom arrays to solve MIS.

  \CiteDesc{pichler2018computational}{arXiv}
  Showed that find the ground state of Rydberg system is NP-complete.
  \Keywords{complexity}

  \CiteDesc{kim2022rydberg}{Nature Physics}
  An early work that demonstrates solving MIS on non-planar graph and high-degree graph using 3D Rydberg atom arrays.
  \Keywords{3D}

  \CiteDesc{dalyac2024graph}{}
  Review of graph problems on atom arrays.
  \Keywords{review}

  \CiteDesc{wurtz2024industry}{arXiv}
  Industry applications of neutral-atom quantum computing solving MIS.

  \CiteDesc{byun2022finding}{PRX Quantum}
  Finding the MIS of platonic graphs using Rydberg atoms.

  \CiteDesc{biedl2006three}{Algorithmica}
  The volume of drawing a graph in 3D space is $\Theta(m\sqrt{n})$.
  \Keywords{3D}
\end{description}

\bibliographystyle{plainnat}
\bibliography{refs}

\end{document}
